\documentclass[10pt,twocolumn,letterpaper]{article}
% PACKAGES ===================================================================================
\usepackage{iccv}
\usepackage{times}
\usepackage{epsfig}
\usepackage{graphicx}
\usepackage{color}
\usepackage{amsmath,amssymb}
\usepackage{algorithm,algpseudocode}
\usepackage{subfig}
\graphicspath{{./figures/}}
\DeclareGraphicsExtensions{.pdf,.jpeg,.png,.jpg}

% Include other packages here, before hyperref.

% If you comment hyperref and then uncomment it, you should delete
% egpaper.aux before re-running latex.  (Or just hit 'q' on the first latex
% run, let it finish, and you should be clear).
\usepackage[pagebackref=true,breaklinks=true,letterpaper=true,colorlinks,bookmarks=false]{hyperref}

% ============================================================================================
% COMMENTS
% ============================================================================================
\newcommand{\stavros}[1]{{\textcolor{red}{[\emph{stavros}: #1]}}}

% ============================================================================================
% GENERAL DEFINITIONS
% ============================================================================================
\usepackage{xspace}
\def\sota{state-of-the-art}
\def\groundtruth{ground truth}
\def\CUB{CUB-200-2011}
\def\eg{\emph{e.g}\onedot} \def\Eg{\emph{E.g}\onedot}
\def\ie{\emph{i.e}\onedot} \def\Ie{\emph{I.e}\onedot}
\def\cf{\emph{c.f}\onedot} \def\Cf{\emph{C.f}\onedot}
\def\etc{\emph{etc}\onedot} \def\vs{\emph{vs}\onedot}
\def\wrt{w.r.t\onedot} \def\dof{d.o.f\onedot}
\def\etal{\emph{et al}\onedot}
\def\kmeans{\emph{k}-means}
\def\cpp{C\texttt{++}}
\def\matlab{MATLAB}
\def\matconvnet{MatConvNet}

% Chapters, Sections, Figures, Equations
\newcommand{\refchap}[1]{Chapter~\ref{#1}}
\newcommand{\refsec}[1]{Section~\ref{#1}}
\newcommand{\reffig}[1]{Figure~\ref{#1}}
\newcommand{\refeq}[1]{Equation~\ref{#1}}
\newcommand{\refalg}[1]{Algorithm~\ref{#1}}
\newcommand{\reftab}[1]{Table~\ref{#1}}
\newcommand{\refapp}[1]{Appendix~\ref{#1}}

% ============================================================================================
% MATH DEFINITIONS 
% ============================================================================================
\newcommand{\pd}[2]{\frac{\partial #1}{\partial #2}}
\newcommand{\mat}[1]{\mathbf{#1}} 
\renewcommand{\vec}[1]{\mathbf{#1}}
\newcommand{\set}[1]{\mathcal{#1}}
\newcommand{\inprod}[1]{{\left< \, #1 \, \right>}}
\newcommand{\R}{\mathbb{R}}
\def\abs{\operatorname{abs}}
\DeclareMathOperator*{\argmin}{arg\,min}
\DeclareMathOperator*{\argmax}{arg\,max}
\newcommand{\p}[1]{\vec{p}_{#1}} 	 % point in space domain with subscript
\newcommand{\f}[1]{\vec{f}_{#1}} 	 % point in space domain with subscript
\newcommand{\g}[1]{\vec{g}_{#1}} 	 % point in space domain with subscript
\newcommand{\norm}[1]{\left \lVert #1 \right \rVert} % norm
%\newcommand{\norm}[1]{|| #1 ||} % norm


\def\iccvPaperID{1142} % *** Enter the ICCV Paper ID here
\def\httilde{\mbox{\tt\raisebox{-.5ex}{\symbol{126}}}}

\begin{document}
\title{AMAT: Medial Axis Transform for Natural Images \\ Supplemental Material}
\maketitle
%\thispagestyle{empty}

\section{Introduction}\label{sec:introduction}
In this supplemental material we include details and qualitative results that we could not include in the main submission due to the limited space.
\stavros{Complete depending on the material we add. Don't forget to include the video.}


\section{Complexity Analysis}\label{sec:complexity}
First, we re-define some useful quantities to make the supplemental material self-contained.
We consider a finite set of radii $r\in \scales : \{r_1,\ldots,r_R \}$, and a disk of radius $r_j$, centered at point $\p{i}\in X^I$,
is denoted as $D_{\p{i},r_j} = D_{ij}$.
The most demanding step in our method is the computation of the disk costs
\begin{equation}
c_{ij} = \sum_k \sum_l \norm{\f{ij} - \f{kl} }^2 \quad \forall k,l: D_{kl} \subset D_{ij}.
\label{eq:diskcost}
\end{equation}
This quantity must be computed at all location-radius combinations, and for all the contained disks, of radii $r\in\scales$.

For simplicity, we make all derivations in the continuous domain, but they can be readily extended to the discrete domain.
Let $r\in(0,R]$; given a disk of radius $r'$ at point $\p{}$, the number of all fully contained disks of radii $r<r'$ is
\begin{equation}
N_d^{r'} = \int_{0}^{r'} 2 \pi r (r'-r) dr,\label{eq:ndr}
\end{equation}
and summing over all radii $r'$ we have that the total number of disks at $\p{}$ is
\begin{equation}
N_d = \int_0^R N_d^{r'}dr'.\label{eq:nd}
\end{equation}
Putting together~\refeq{eq:ndr} and~\refeq{eq:nd} we get:
\begin{align}
N_d & = 2\pi \int_0^{R} \int_0^{r'} r (r'-r) dr dr' \\
  & = 2\pi \int_0^{R} \int_0^{r'} rr'-r^2 dr dr' \\
  & = 2\pi \int_0^{R} \left. r'\frac{r^2}{2}-\frac{r^3}{3} \right \rvert_{0}^{r'} dr' \\
  & = 2\pi \int_0^{R} \frac{(r')^3}{6} dr' \\
  & = 2\pi \int_0^{R} \frac{(r')^3}{6} dr' = \frac{\pi}{12} R^4
\end{align}





\section{Qualitative Results}\label{sec:qualitative}
In this section we present additional qualitative results of medial axes computed with our method, and the respective reconstructions.
We also include a video showing the execution of the greedy algorithm on a test image.
\stavros{Add more details about the various details shown in the video}


\begin{figure*}[t]
\centering
\def\imgw{0.245}

\def\img_id{85048}
\includegraphics[width=\imgw \textwidth]{\img_id_resized.jpg}
\includegraphics[width=\imgw \textwidth]{\img_id_axes_simplified.pdf}
\includegraphics[width=\imgw \textwidth]{\img_id_branches_simplified.pdf}
\includegraphics[width=\imgw \textwidth]{\img_id_gt_skel.pdf}

\def\img_id{295087}
\includegraphics[width=\imgw \textwidth]{\img_id_resized.jpg}
\includegraphics[width=\imgw \textwidth]{\img_id_axes_simplified.pdf}
\includegraphics[width=\imgw \textwidth]{\img_id_branches_simplified.pdf}
\includegraphics[width=\imgw \textwidth]{\img_id_gt_skel.pdf}

\def\img_id{145086}
\includegraphics[width=\imgw \textwidth]{\img_id_resized.jpg}
\includegraphics[width=\imgw \textwidth]{\img_id_axes_simplified.pdf}
\includegraphics[width=\imgw \textwidth]{\img_id_branches_simplified.pdf}
\includegraphics[width=\imgw \textwidth]{\img_id_gt_skel.pdf}

\def\img_id{101087}
\includegraphics[width=\imgw \textwidth]{\img_id_resized.jpg}
\includegraphics[width=\imgw \textwidth]{\img_id_axes_simplified.pdf}
\includegraphics[width=\imgw \textwidth]{\img_id_branches_simplified.pdf}
\includegraphics[width=\imgw \textwidth]{\img_id_gt_skel.pdf}

\def\img_id{54082}
\includegraphics[width=\imgw \textwidth]{\img_id_resized.jpg}
\includegraphics[width=\imgw \textwidth]{\img_id_axes_simplified.pdf}
\includegraphics[width=\imgw \textwidth]{\img_id_branches_simplified.pdf}
\includegraphics[width=\imgw \textwidth]{\img_id_gt_skel.pdf}
\caption{\textbf{Medial axes}. From left to right: Input image, AMAT medial axes, medial branches (color-coded), ground-truth skeletons.
Axis color indicates the respective encodings $\f{}$, and black is used for unused points.}
\label{fig:axes}
\end{figure*}



\begin{figure*}[t]
\centering
\def\imgw{0.195}

\def\img_id{3096}
\includegraphics[width=\imgw \textwidth]{\img_id_resized.jpg}
\includegraphics[width=\imgw \textwidth]{\img_id_rec_mil.png}
\includegraphics[width=\imgw \textwidth]{\img_id_rec_gtseg.png}
\includegraphics[width=\imgw \textwidth]{\img_id_rec_gtskel.png}
\includegraphics[width=\imgw \textwidth]{\img_id_rec_amat.png}

\def\img_id{300091}
\includegraphics[width=\imgw \textwidth]{\img_id_resized.jpg}
\includegraphics[width=\imgw \textwidth]{\img_id_rec_mil.png}
\includegraphics[width=\imgw \textwidth]{\img_id_rec_gtseg.png}
\includegraphics[width=\imgw \textwidth]{\img_id_rec_gtskel.png}
\includegraphics[width=\imgw \textwidth]{\img_id_rec_amat.png}

\def\img_id{119082}
\includegraphics[width=\imgw \textwidth]{\img_id_resized.jpg}
\includegraphics[width=\imgw \textwidth]{\img_id_rec_mil.png}
\includegraphics[width=\imgw \textwidth]{\img_id_rec_gtseg.png}
\includegraphics[width=\imgw \textwidth]{\img_id_rec_gtskel.png}
\includegraphics[width=\imgw \textwidth]{\img_id_rec_amat.png}

\def\img_id{295087}
\includegraphics[width=\imgw \textwidth]{\img_id_resized.jpg}
\includegraphics[width=\imgw \textwidth]{\img_id_rec_mil.png}
\includegraphics[width=\imgw \textwidth]{\img_id_rec_gtseg.png}
\includegraphics[width=\imgw \textwidth]{\img_id_rec_gtskel.png}
\includegraphics[width=\imgw \textwidth]{\img_id_rec_amat.png}

\def\img_id{54082}
\includegraphics[width=\imgw \textwidth]{\img_id_resized.jpg}
\includegraphics[width=\imgw \textwidth]{\img_id_rec_mil.png}
\includegraphics[width=\imgw \textwidth]{\img_id_rec_gtseg.png}
\includegraphics[width=\imgw \textwidth]{\img_id_rec_gtskel.png}
\includegraphics[width=\imgw \textwidth]{\img_id_rec_amat.png}

\def\img_id{101087}
\includegraphics[width=\imgw \textwidth]{\img_id_resized.jpg}
\includegraphics[width=\imgw \textwidth]{\img_id_rec_mil.png}
\includegraphics[width=\imgw \textwidth]{\img_id_rec_gtseg.png}
\includegraphics[width=\imgw \textwidth]{\img_id_rec_gtskel.png}
\includegraphics[width=\imgw \textwidth]{\img_id_rec_amat.png}

\caption{\textbf{Image reconstruction}. From left to right: Input image, MIL~\cite{tsogkas2012learning}, GT-seg, GT-skel, AMAT.}
\label{fig:experiments:reconstruction}
\end{figure*}





{\small
\bibliographystyle{ieee}
\bibliography{iccv2017.bib}
}

\end{document}